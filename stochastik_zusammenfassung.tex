\documentclass[11pt,a4paper,ngerman]{article}

% BASICS
\usepackage[ngerman]{babel}
\usepackage[utf8x]{inputenc}
\usepackage{parskip}
\usepackage{amsfonts}
\usepackage{amsmath}
\usepackage{amssymb}
\usepackage{enumerate}
\usepackage{float}
\usepackage{graphicx}
\usepackage{caption}
\usepackage{multicol}
\usepackage[left=3cm,right=3cm,top=2cm,bottom=3.5cm,includeheadfoot]{geometry}
\usepackage{url}
\usepackage{hyperref}

% ZAHLEN
\newcommand{\N}{\mathbb{N}}
\newcommand{\Z}{\mathbb{Z}}
\newcommand{\Q}{\mathbb{Q}}
\newcommand{\R}{\mathbb{R}}
\newcommand{\C}{\mathbb{C}}

% MENGEN
\newcommand{\Po}{\mathfrak{P}}
\newcommand{\compl}{^{\text{\tiny C}}}
\newcommand{\pd}{\underline{\operatorname{p.d.}}}

% STOCHASTIK
\renewcommand{\O}{\Omega}
\newcommand{\A}{\mathcal{A}}
\renewcommand{\Pr}{\mathbb{P}}
\newcommand{\E}{\mathbb{E}}
\newcommand{\Var}{\operatorname{Var}}
\newcommand{\Cov}{\operatorname{Cov}}
\newcommand{\F}{\mathcal{F}}
\newcommand{\e}{\mathcal{E}}
\renewcommand{\L}{\mathcal{L}}
\newcommand{\B}{\mathfrak{B}}
\renewcommand{\o}{\omega}
\newcommand{\iid}{\operatorname{i.i.d}}

% ANALYSIS
\newcommand{\limfty}[1]{\lim\limits_{#1\to\infty}}
\renewcommand{\max}{\operatorname{max}}
\renewcommand{\min}{\operatorname{min}}

% SONSTIGES
\newcommand{\U}{\texttt{Übung}}

% CENTERED COLUMNS WITH FIX WIDTH
\usepackage{array}
\newcolumntype{P}[1]{>{\centering\arraybackslash}p{#1}}

% META
\date{April 2019}
\author{Maximilian Reif}
\title{Zusammenfassung Einführung in die Stochastik}
\hypersetup{pdftex,
            pdfauthor={Maximilian Reif},
            pdftitle={Zusammenfassung Einführung in die Stochastik},
            pdfsubject={},
            pdfkeywords={},
            pdfproducer={},
            pdfcreator={},
            pdfpagemode=UseOutlines
}

%-------------------------------------------------------------------------------

\begin{document}

\begin{titlepage}
    \ \newline\newline\newline\newline\newline

  \begin{center}

  \huge Zusammenfassung zur\\
  \Huge\textbf{Einführung in die Stochastik}\\
  \huge im Wintersemester 17/18\\
  \normalsize

  \vspace{1cm}
  \begin{tabular}[b]{l|l}
  \textbf{Autor}         & Maximilian Reif
    \texttt{\href{mailto:reifmaxi@fim.uni-passau.de}{<reifmaxi@fim.uni-passau.de>}} \\
  \textbf{Version}       & April 2019 \\
  \textbf{GitHub}        & \url{https://github.com/lordreif/stochastik}
  \end{tabular}
  \vspace{1cm}

  \end{center}

  \tableofcontents

  % FIM logo
  \begin{figure}[b]
  \centering
  \includegraphics[width=0.6\textwidth]{\string~/Templates/img/fim.png}
  \end{figure}


\end{titlepage}

\setcounter{page}{1}
\section{Grundlagen}

\textbf{$\sigma$-Algebra}
\begin{enumerate}
\item $\O\in\A$
\item $A\in\A \Rightarrow A\compl\in\A$
\item $A_1,A_2,\ldots\in\A \Rightarrow \bigcup_{i=1}^{\infty} A_i \in\A$
\end{enumerate}

\textbf{Wahrscheinlichkeitsmaß} $\Pr\!: \A \rightarrow [0,1]$
\begin{enumerate}
\item $\Pr(\O) = 1$
\item $A_1,A_2,\ldots\in\A\ \pd \Rightarrow
  \Pr(\bigcup_{i=1}^{\infty} A_i) = \sum_{i=1}^{\infty} \Pr(A_i)$
  \hfill\textit{($\sigma$-Additivität)}
\end{enumerate}

Daraus folgt für $A,B,A_1,A_2,\ldots\in\A$:
\begin{itemize}
\item $A \subset B \Rightarrow \Pr(B) = \Pr(A) + \Pr(B \setminus A)$
\item $A \subset B \Rightarrow \Pr(A) \leq \Pr(B)$ \hfill\textit{(Monotonie)}
\item $\Pr(A\compl) = 1 - \Pr(A)$
\item $\Pr(A \cup B) = \Pr(A) + \Pr(B) - \Pr(A \cap B)$
\item $\Pr(\bigcup_{i=1}^{\infty} A_i) \leq \sum_{i=1}^{\infty} \Pr(A_i)$
  \hfill\textit{($\sigma$-Subadditivität)}
\item $A_1 \subset A_2 \subset \cdots \Rightarrow
  \Pr(\bigcup_{i=1}^{\infty} A_i) = \limfty{i} \Pr(A_i)$
  \hfill(\textit{$\sigma$-Stetigkeit von unten)}
\item $A_1 \supset A_2 \supset \cdots \Rightarrow
  \Pr(\bigcap_{i=1}^{\infty} A_i) = \limfty{i} \Pr(A_i)$
  \hfill(\textit{$\sigma$-Stetigkeit von oben)}
\end{itemize}

\textbf{Bedingte Wahrscheinlichkeit}\\
Falls $\Pr(B) > 0$, dann
$\Pr(\ \cdot \mid B) := \dfrac{\Pr(\ \cdot\ \cap B)}{\Pr(B)}$
W'Maß auf $\A$ mit $\Pr(B \mid B) = 1$.

Für $A,B,A_1,A_2,\ldots\in\A$ mit
$\Pr(A),\Pr(B), \Pr(\bigcap_{i=1}^{n-1} A_i) > 0$ gilt:

\begin{itemize}
\item $\Pr(B \mid A) = \dfrac{\Pr(A \mid B) \cdot \Pr(B)}{\Pr(A)}$
\item $\Pr(\bigcap_{i=1}^{n} A_i) = \Pr(A_1) \cdot \Pr(A_2 \mid A_1) \cdots
  \Pr(A_1 \cap \cdots \cap A_n) \cdot \Pr(A_n \mid A_1 \cap \cdots \cap A_{n-1})$
\end{itemize}

\textbf{Formel der totalen Wahrscheinlichkeit}\\
Für $B_1,B_2,\ldots,B_n\in\A\ \pd$ mit $\Pr(B_i) > 0$ für alle
$i\in\{1,\ldots,n\}$ und $\bigcup_{i=1}^{n} B_i = \O$ gilt:
\[
  \Pr(A) = \sum_{i=1}^{n} \Pr(A \mid B_i) \cdot \Pr(B_i) \text{ für alle } A\in\A.
\]

\newpage

\textbf{Formel von Bayes}\\
Für $B_1,B_2,\ldots,B_n\in\A\ \pd$ mit $\Pr(B_i) > 0$ für alle
$i\in\{1,\ldots,n\}$ und $\bigcup_{i=1}^{n} B_i = \O$ gilt:
\[
  \Pr(B_i \mid A)=
  \frac{\Pr(A \mid B_i) \cdot \Pr(B_i)}{\sum\limits_{j=1}^{n} \Pr(A \mid B_j)
  \cdot \Pr(B_j)} \text{ für alle } A\in\A \text{ mit } \Pr(A) > 0.
\]

\textbf{Unabhängigkeit}\\
$A,B\in\A$ unabhängig $:\Leftrightarrow \Pr(A \cap B) = \Pr(A) \cdot \Pr(B)
\Leftrightarrow P(A \mid B) = P(A)$ für $\Pr(B) > 0$

$A_1,\ldots,A_n\in\A$ unabhängig $\Leftrightarrow \Pr(\bigcap_{i=1}^{n} A_i) = $
$\prod_{i=1}^{n} \Pr(A_i) \Rightarrow $ paarweise Unabhängigkeit

\textbf{Diskrete Modelle}\\


\begin{table}[h]
\hspace{-2.5em}
\begin{tabular}{l|l|l|}
\cline{2-3}
                                       & \begin{tabular}[c]{@{}l@{}} mit Berücksichtigung\\ der Reihenfolge\end{tabular} & \begin{tabular}[c]{@{}l@{}}ohne Berücksichtigung\\ der Reihenfolge\end{tabular}                                                                   \\ \hline
\multicolumn{1}{|l|}{mit Zurücklegen}  & \begin{tabular}[c]{@{}l@{}}$\O=N^k$,\\ $|\O| = n^k$\end{tabular}   & \begin{tabular}[c]{@{}l@{}}$\O = \{\o\in N^k\mid1\leq\o_1\leq\ldots\leq\o_n\leq n\}$,\\ $|\O| = \binom{n+k-1}{k}$\end{tabular} \\ \hline
\multicolumn{1}{|l|}{ohne Zurücklegen} & \begin{tabular}[c]{@{}l@{}}$\O=\{\o\in N^k \mid \o_i\neq\o_j \text{ für } i\neq j\} $\\ $|\O|=\frac{n!}{(n-k)!}$\end{tabular}  & \begin{tabular}[c]{@{}l@{}}$\O=\{\o\in N^k\mid 1\leq\o_1<\ldots<\o_k\leq n\}$,\\ $|\O|=\binom{n}{k}$\end{tabular}              \\ \hline
\end{tabular}
\end{table}

Für $n,k\in\N$ mit $k \leq n$ gilt
$\binom{n}{k} := \frac{n!}{k!(n-k)!}$,
$\binom{n}{k-1} + \binom{n}{k} = \binom{n+1}{k}$,
$\binom{n}{k} = \binom{n}{n-k}$.

\section{Integration}

\textbf{Lebesgue-Maß} $\lambda_d\!: \B_d \to [0,\infty]$
\begin{itemize}
\item Für $B_1,B_2,\ldots\in\B_d\ \pd$:\:
  $\lambda_d(\bigcup_{i=1}^{\infty} B_i) = \sum_{i=1}^{\infty} \lambda_d(B_i)$.

\item Für $a_1,b_1,\ldots,a_d,b_d\in\R$ mit $a_i \leq b_i$:\:\:
  \mbox{$\lambda_d([a_1,b_1] \times\cdots\times [a_d,b_d]) =
  \prod_{i=1}^{d} (b_i-a_i)$.}

\item Für $a\in\R^d$, $Q\in\mathrm{O}(d)$, $A\in\B_d$:
  $\lambda_d(a+Q(A)) = \lambda_d(A)$.

\item $\forall x\in\R^d\!: \lambda_d(\{x\}) = 0$

\item $\lambda_d$ ist $\sigma$-subadditiv, $\sigma$-stetig von unten
  (nicht von oben!) und monoton
\end{itemize}

\textbf{Borel-messbare Funktionen}
\begin{itemize}
\item $f\!: \R^d\to\R$ \textit{Borel-messbar}
  $:\Leftrightarrow \forall B\in\B_1\!: \{f \in B\} \in\B_d$

\item $f\!: \R^d\to\overline{\R}$ \textit{Borel-messbar}
  $:\Leftrightarrow \forall B\in\B_1\!: \{f \in B\} \in\B_d$
  \mbox{und $\{f = \infty\} \in\B_d$}

\item $\F_d := \{f\!: \R^d \to \overline{\R} \mid f \text{ ist Borel-messbar}\}$,
  $\F_d^+ := \{f\in\F_d \mid f \geq 0\}$

\item $\forall B\in\B_d\!: \chi_B\in\F_d^+$

\item $f$ stetig $\Rightarrow\ f\in\F_d$

\item $f,g\in\F_d \Rightarrow a \cdot f + g, f \cdot g, f/g,
  \max(f,g)\in\F_d$ \mbox{falls wohldefiniert ($a\in\overline{\R}$)}
\end{itemize}

\textbf{Elementare Funktionen}
\begin{itemize}

\item Abbildung $\e\!: \R^d \to \overline{\R}$ \textit{elementar} $:\Leftrightarrow$\\
  $\exists b_1,\ldots,b_n\in\overline{\R} \wedge \exists B_1,\ldots,B_n\in\B_d\
  \pd$ mit $\e(x) = \sum_{i=1}^{n} b_i \cdot \chi_{B_i}(x)$ für alle $x\in\R^d$

\item $\e_d := \{f\!: \R^d \to \overline{\R} \mid f \text{ ist elementar}\}$,
  $\e_d^+ := \{f\in\e_d \mid f\geq 0\}$

\item $\e_d \subset \F_d$
\end{itemize}

\textbf{Lebesgue-Integral}
\begin{itemize}
\item Für $f\in\e_d^+$:
  $\int f~d\lambda_d := \sum_{i=1}^{n} b_i \cdot \lambda_d(B_i) \in [0,\infty]$

\item Das Lebesgue-Integral ist definiert für alle $f\in\F_d^+$.

\item $\int_B f~\lambda_d := \int (f\cdot \chi_B)~d\lambda_d$ für $B\in\B_d$

\item Das Lebesgue-Integral ist monoton und linear.

\item Für auf $[a,b]$ beschränktes $f \in \F_1$ mit
  \mbox{$\lambda(\{x\in[a,b] \mid f \text{ unstetig in }x\}) = 0$ gilt:}
  \[
    \int_{[a,b]} f~d\lambda =
    \underbrace{\int_{a}^{b}f(x)~dx}_{\text{Riemann-Integral}}
  \]

\item $\forall A\in\B_d\!: \int_A f~d\lambda_d = \int_A g~d\lambda_d$
  $\Leftrightarrow \lambda_d(\{x\in\R^d \mid f(x) \neq g(x)\}) = 0$
\end{itemize}

\textbf{Satz von der monotonen Konvergenz}\\
Für jede monoton wachsende Folge $0 \leq f_1 \leq f_2 \leq \cdots$
in $\F_d^+$ gilt
\[
  \limfty{i}\int f_i~d\lambda_d =
  \int\limfty{i} f_i~d\lambda_d \in [0,\infty].
\]

\textbf{Satz von der dominierten Konvergenz}\\
Für $f,f_1,f_2,\ldots\in\F_d$ mit $\limfty{i} f_i = f$
und $g\in\F_d$ integrierbar mit $|f_i| \leq g$ für alle $i\in\N$ gilt
\[
  \int f~d\lambda_d = \limfty{i} \int f_i~d\lambda_d.
\]
Insbesondere für $f\in\F_1$ integrierbar oder $f\in\F_1^+$:
\[
  \int_\R f~d\lambda = \limfty{i} \int_{[i,i]} f~d\lambda.
\]


\textbf{Satz von Fubini}\\
Für $f\in\F_d$ integrierbar oder $f\in\F_d^+$ und
$B = B_1 \times\cdots\times B_d \subset \B_1^d$ gilt
\[
  \int_B f~d\lambda_d = \int_{B_{i_1}} \cdots \int_{B_{i_d}} f(x_1,\ldots,x_d)~
  d\lambda(x_{i_d}) \cdots d\lambda(x_{i_1})
\]
für jede Permutation $(i_1,\ldots,i_d)$ von $(1,\ldots,d)$.

\textbf{Substitutionsregel} (Riemann-Integration)\\
Für $a,b\in\R$ mit $a<b$, $g\!: [a,b] \to I$ stetig differenzierbar,
$f\!: I\to\R$ stetig gilt
\[
  \int_a^b f(g(y)) \cdot g'(y)~dy = \int_{g(a)}^{g(b)}f(x)~dx.
\]

\textbf{Partielle Integration} (Riemann-Integration)\\
Für $a,b\in\R$ mit $a<b$ und $f,g\!: [a,b]\to\R$ stetig differenzierbar gilt
\[
  \int_a^b f'(x) \cdot g(x)~dx
  = f(x) \cdot g(x) \Big\vert_a^b - \int_a^b f(x) \cdot g'(x)~dx.
\]

\section{Zufallsvariablen}
\textbf{Reellwertige Zufallsvariable}\\
$X\!: \O\to\R$ ist \textit{reellwertige Zufallsvariable} auf $(\O,\A,\Pr)$,
wenn
\[
  \forall I \text{ Intervall}\!: \{\o\in\O \mid X(\o) \in I\} \in \A.
\]
Falls $\A = \Po(\O)$ ist jedes $X\!: \O\to\R$ Zufallsvariable.

\textbf{Verteilungsfunktion}\\
Die \textit{Verteilungsfunktion}
\[
  F_X\!: \R \to [0,1],\ x \mapsto \Pr(\{X \leq x\})
\]
einer Zufallsvariablen ist monoton wachsend, rechtsseitig stetig und es gilt\\
$\limfty{x} F_X(x) = 1$,
$\lim\limits_{x\to -\infty} F_X(x) = 0$,
$F_X(x) - F_X(x\textbf{-}) = \Pr(\{X = x\})$.

\textbf{Reellwertiger Zufallsvektor}\\
$X = (X_1,\ldots,X_d)\!: \O\to\R^d$ ist \textit{reellwertiger Zufallsvektor},
wenn jede Komponente $X_i$ reellwertige Zufallsvariable ist.
Es gilt:\\
\[
  X \text{ ist Zufallsvektor } \Leftrightarrow X \text{ ist Borel-messbar.}
\]

\textbf{Verteilung}\\
Die \textit{Verteilung} eines $d$-dimensionalen Zufallsvektors $X$:
\[
  \Pr_X\!: \B_d \to [0,1],\ B \mapsto \Pr(\{X \in B\})
\]
ist ein Wahrscheinlichkeitsmaß auf $\B_d$.

\textbf{Randverteilung}\\
Zu einem Zufallsvektor $X = (X_1,\ldots,X_d)$ heißen ($i\in\{1,\ldots,d\}$)
\[
  \Pr_{X_i}\!: \B_1 \to [0,1],\ B \mapsto \Pr(\{X_i \in B\})
\]
die \textit{(eindimensionalen) Randverteilungen} von $X$.

\textbf{Wahrscheinlichkeitsfunktion}\\
Sei $\O$ abzählbar.
Eine Funktion $f\!: \O \to [0,\infty)$ heißt
\textit{Wahrscheinlichkeitsfunktion}, falls
\[
  \sum_{\o\in\O} f(\o) = 1.
\]

\textbf{Wahrscheinlichkeitsdichte}\\
Eine Funktion $f\in\F_d^+$ heißt \textit{Wahrscheinlichkeitsdichte},
falls
\[
  \int_{\R^d} f(x)~d\lambda(x) = 1.
\]

\textbf{Bedeutung von W'funktionen/-dichten}\\
Wahrscheinlichkeitsfunktionen/-dichten definieren Wahrscheinlichkeitsmaß auf
$\A$ durch
\[
  \Pr(A) = \sum_{\o \in A} f(\o) \text{ bzw. } \Pr(A) = \int_A f(x)~d\lambda(x)
  \quad \text{für } A\in\A.
\]

Zu jeder Wahrscheinlichkeitsfunktion/-dichte $f$ gibt es $(\O,\A,\Pr)$, sodass
darauf eine (diskrete/absolut stetige) Zufallsvariable $X$ mit
$f_X = f$ existiert.

\textbf{Diskrete Zufallsvariablen}\\
Eine Zufallsvariable heißt \textit{diskret}, wenn $\Pr(\{X \in D\}) = 1$
für ein abzählbares $D\subset\R$ gilt.
Das kleinste solche $D =: D_X$ heißt \textit{Träger} der Zufallsvariablen $X$.\\
Es gilt:
$\underbrace{(\O,\A,\Pr) \text{ diskret}}_{\text{also $\O$ abzählbar}}$
$\Rightarrow X(\O)$ abzählbar $\Rightarrow X$ diskret.

\textbf{Diskrete Zufallsvektoren}\\
Ein $d$-dimensionaler Zufallsvektor $X = (X_1,\ldots,X_d)$ heißt \textit{diskret},
wenn es ein abzählbares $D\subset\R^d$ mit $\Pr(\{X \in D\}) = 1$ gibt.\\
Dann ist $D_X = \{x\in\R^d \mid \Pr(\{X=x\})>0\}
\overset{\text{i.A.}}{\neq} D_{X_1} \times\cdots\times D_{X_d}$
der Träger von $X$.\\
Es gilt: $X$ diskret $\Leftrightarrow \forall i\in\{1,\ldots,d\}\!: X_i$ diskret.\\
Für $h:\R^d\to\R$ Borel-messbar ist $h(X)$ diskret mit $D_{h(X)} = h(D_X)$.

\textbf{Absolut stetige Zufallsvektoren}\\
Ein Zufallsvektor $X$ heißt \textit{absolut stetig verteilt} falls die
Verteilung $\Pr_X$ eine Dichte $f_X$ besitzt.

\textbf{Unabhängigkeit von Zufallsvariablen}\\
Eine Folge $(X_i)_{i \in I}$ von Zufallsvariablen heißt \textit{unabhängig} wenn
für jede Folge $(J_i)_{i \in I}$ von Intervallen die Folge der Ereignisse
$(\{X_i \in J_i\})_{i \in I}$ unabhängig ist.
Wegen:
\[
  \forall x\in\R\!: \{X \leq x\} \in \A
  \Leftrightarrow \forall I \text{ Intervall}\!: \{X \in I\} \in \A
\]
ist $(X_i)_{i \in I}$ genau dann unabhängig, wenn für alle endlichen Mengen
$\emptyset \neq \tilde{I} \subset I$ und
\mbox{$(x_i)_{i\in\tilde{I}}\subset\R$} gilt:
\[
  \Pr(\bigcap_{i\in\tilde{I}} \{X_i \leq x_i\})=
  \prod_{i\in\tilde{I}} \Pr(\{X_i \leq x_i\}).
\]
Die Unabhängigkeit einer Folge $(X_i)_{i \in I}$ impliziert die paarweise
Unabhängigkeit von $X_i,X_j$ für $i,j \in I$ mit $i \neq j$.

Beachte auch (\ref{unabhaengig_e}) und (\ref{unabhaengig_kor}).

Es gilt: $X,X$ unabhängig $\Leftrightarrow \exists c\in\R\!: \Pr(\{X=c\}) = 1$ \U

\textbf{Identische Verteilung}\\
Zwei Zufallsvariablen $X,Y$ heißen \textit{identisch verteilt},
wenn $\Pr_X = \Pr_Y \Leftrightarrow F_X = F_Y$.\\
Zwei Zufallsvektoren
$\tilde{X} = (\tilde{X}_1,\ldots,\tilde{X}_d),
\tilde{Y} = (\tilde{Y}_1,\ldots,\tilde{Y}_d)$
heißen \textit{identisch verteilt}, wenn
$\Pr_{\tilde{X}} = \Pr_{\tilde{Y}}$.\\
Es gilt: $\tilde{X},\tilde{Y}$ identisch verteilt
$\Rightarrow \forall i\in\{1,\ldots,d\}\!: \tilde{X}_i,\tilde{Y}_i$ identisch verteilt.

\newpage
\begin{table}[h]
\centering
\caption*{\textbf{Vergleich}}
\begin{tabular}{P{0.45\linewidth} | P{0.45\linewidth}}

\textbf{diskret} & \textbf{absolut stetig} \\

\multicolumn{2}{c}{\textit{W'funktion/Dichte}}  \\

Es existiert W'funktion: & \\
$f_X\!: D_X \to [0,1],\ x \mapsto \Pr(\{X=x\})$  &
$\Pr_X$ besitzt Dichte $f_X$ \\

&\\ %empty row

\multicolumn{2}{c}{\textit{Verteilungsfunktion (eindim.!)}}  \\

$$F_X(x) = \sum\limits_{y \in D_X \cap (-\infty, x]} \Pr(\{X=y\})$$  &
$$F_X(x) = \int_{(-\infty,x]}f_X(y)~d\lambda(y)$$  \\

\multicolumn{2}{c}{\textit{Verteilung}}  \\

$$\Pr_X(A) = \sum\limits_{x\in D_X \cap A} \Pr(\{X=x\})$$  &
$$\Pr_X(A) = \int_{A} f_X(x)~d\lambda(x)$$  \\

\multicolumn{2}{c}{\textit{Randverteilung/Randdichte}}  \\
\multicolumn{2}{c}{$X=(X_1\ldots,X_d)$}  \\

$P_{X_i}(\{x\}) = $
  \mbox{$\Pr(\{X_i=x\} \cap \{(X_1,\ldots,X_d) \in D_X\})$}	&
$f_{X_i}(x) =$
\mbox{$\int_\R\cdots\int_\R f_X(x_1,\ldots,x_{i-1},x,x_{i+1},\ldots,x_d)~
d\lambda(x_1) \cdots d\lambda_{x_d}$}  \\
&\\ %empty row
&
(alles integrieren außer $i$-te Koordinate)  \\
&\\ %empty row

\multicolumn{2}{c}{\textit{$X_1,\ldots,X_d$ unabhängig}}\\

$\forall x_1,\ldots,x_d\in\R\!:$
\mbox{$\Pr(\bigcap_{i=1}^d \{X_i=x_i\}) = \prod_{i=1}^d \Pr(\{X_i=x_i\})$}  &
$\forall B_1,\ldots,B_d\in\B_1:$
\mbox{$\Pr(\bigcap_{i=1}^d \{X_i\in B_i\}) = \prod_{i=1}^d \Pr(\{X_i\in B_i\})$}  \\
&
$\Leftrightarrow$  \\
&
$\forall B_1,\ldots,B_d\in\B_1:$
\mbox{$\Pr_X(B_1\times\cdots\times B_d) = \prod_{i=1}^d \Pr_{X_i}(B_i)$}  \\
&\\ %empty row
&
Gemeinsame Verteilung $P_X$ von $X_1,\ldots,X_d$ ist also Produkt der
Randverteilungen $P_{X_{i\in\{1,\ldots,d\}}}$  \\
&\\ %empty row

\multicolumn{2}{c}{\textit{$X,Y$ identisch verteilt (eindim.!)}}  \\

$D_X = D_Y$ und $f_X = f_Y$ &
$P_X=P_Y$  \\
$\forall z\in\R\!: \Pr(\{X=z\}) = \Pr'(\{Y=z\})$  &
$\forall A\in\B_1\!: \Pr(\{X \in A\}) = \Pr'(\{Y \in A\})$ \\

\end{tabular}
\end{table}

\section{Erwartungswert \& Varianz}

\textbf{Integriebare Zufallsvariablen}
\begin{itemize}
\item Diskretes $X$ mit Träger $D_X$ \textit{integrierbar}
  $:\Leftrightarrow \sum_{x \in D_X} |x| \cdot \Pr(\{X=x\}) < \infty$ \\
  Diskrete Zufallsvariablen mit \underline{endlichem} Träger sind immer
  integrierbar!

\item  Absolut stetige Zufallsvariable $X$ mit Dichte $f_X$ \textit{integrierbar}
  \mbox{$:\Leftrightarrow \int_\R |x| \cdot f_X(x)~d\lambda(x) < \infty$}

\item $\L_1 := \{X \text{ integrierbar}\}$ ist ein Vektorraum.

\item \textit{Erwartungswert $\E(X)$} für diskretes $X\in\L_1$:
  $\E(X) := \sum_{x \in D_X} x \cdot \Pr(\{X=x\}) \in\R$

\item \textit{Erwartungswert $\E(X)$} für absolut stetiges $X\in\L_1$:
  $\E(X) := \int_\R x\cdot f_X~d\lambda(x) \in\R$

\item Der Erwartungswert ist linear und monoton.

\item \textbf{Transformationssatz}: Für $X$ $d$-dimensionaler Zufallsvektor
  diskret/absolut stetig und \mbox{$h\!:\ \R^d\to\R$ Borel-messbar} gilt
  \[
    h(X)\in\L_1 \Leftrightarrow
    \begin{cases}
    \sum_{x \in D_X} |h(x)| \cdot \Pr(\{X=x\}) < \infty      & X \text{ diskret}  \\
    \int_{\R^d} |h(x)| \cdot f_X(x)~d\lambda_d(x) < \infty   & X \text{ absolut stetig}
    \end{cases}.
  \]
  Gegebenenfalls
  \[
  	\E(h(X)) =
    \begin{cases}
    \sum_{x \in D_X} h(x) \cdot \Pr(\{X=x\}) < \infty			& X \text{ diskret} \\
    \int_{\R^d} h(x) \cdot f_X(x)~d\lambda_d(x) < \infty  & X \text{ absolut stetig}
    \end{cases}.
  \]

\item $X,Y\in\L_1$ unabhängig
  $\Rightarrow X \cdot Y \in\L_1$ mit $\E(X \cdot Y) = \E(X) \cdot \E(Y)$
  \marginpar{\vspace{-2.2em}\begin{equation}\label{unabhaengig_e}\end{equation}}
\end{itemize}
\hspace{3em}

\textbf{Quadratisch integrierbare Zufallsvariablen}
\begin{itemize}
\item $X$ \textit{quadratisch integrierbar} $:\Leftrightarrow X^2\in\L_1$

\item $\L_2 := \{X \text{ quadratisch integrierbar}\}$ ist Untervektorraum von $\L_1$

\item Für $X$ diskret mit Träger $D_X$ bzw. absolut stetig mit Dichte $f_X$ gilt:
  \[
    X\in\L_2\Leftrightarrow
    \begin{cases}
    \sum_{x\ in D_X} x^2 \cdot \Pr(\{X=x\}) < \infty			& \text{X diskret} \\
    \int_{\R^d} x^2\cdot f_X(x)~d\lambda_d(x) < \infty		& \text{X absolut stetig}
    \end{cases}.
  \]
Gegebenenfalls ist $\E(X^2)$ durch obige(s) Summe/Integral gegeben.

\item Für $X\in\L_2$ ist $\Var(X) := \E((X-\E(X))^2)$ die \textit{Varianz} von $X$.

\item $\Var(X) = \E(X^2) - (\E(X))^2$,
  \quad $\Var(\alpha \cdot X + \beta) = \alpha^2\Var(X)$

\item \textbf{Tschebyschev-Ungleichung}. Für $X\in\L_2, \varepsilon > 0$ gilt:
  \[
    \Pr(|X-\E(X)| \geq \varepsilon) \leq \frac{1}{\varepsilon^2} \cdot \Var(X)
  \]
  sowie
  \[
    \Var(X) = 0 \Leftrightarrow \Pr(\{X=\E(x)\}) = 1.
  \]

\item \textbf{Formel von Bienaymé}. Für $X_1,\ldots,X_n\in\L_2$ unabhängig gilt:
  \[
    \Var(\sum_{i=1}^n X_i) = \sum_{i=1}^n \Var(X_i).
  \]

\item $X,Y\in\L_2 \Rightarrow X \cdot Y \in\L_1$

\item $\overline{\L_2}$ ist Hilbertraum mit $\langle X,Y \rangle := \E(X \cdot Y)$
\end{itemize}
\hspace{3em}

\textbf{Kovarianz}
\begin{itemize}
\item Für $X,Y\in\L_2$ ist die \textit{Kovarianz} definiert durch
 \mbox{$\Cov(X,Y) := \E\big( (X-\E(X) \cdot (Y-\E(Y)) \big)$}

\item Für $X,Y\in\L_2$ gilt $\Cov(X,Y) = \E(X \cdot Y) - \E(X) \cdot \E(Y)$

\item $X,Y$ heißen \textit{unkorreliert} wenn $\Cov(X,Y) = 0$.\\
  Es gilt: $X,Y$ unabhängig $\Rightarrow X,Y$ unkorreliert
  \marginpar{\vspace{-3.45em}\begin{equation}\label{unabhaengig_kor}\end{equation}}

\item Für $X,Y$ mit $\Var(X),\Var(Y)>0$ ist
  \[
    \rho(X,Y) := \frac{\Cov(X,Y)}{\sqrt{\Var(X)\cdot\Var(Y)}} =
    \cos(\sphericalangle (X,Y))
  \]
  der \textit{Korrelationskoeffizient} von $X$ und $Y$.

\item \textbf{Cauchy-Schwarz}: $X,Y\in\L_2
  \Rightarrow |\E(X \cdot Y)| \leq \sqrt{\E(X^2) \cdot \E(Y^2)}$
\end{itemize}

\section{Verteilungen}

\textbf{Bernoulli-Verteilung} $X\sim\mathbf{B}(1,p)$ mit $p \in [0,1]$
\begin{itemize}
\item $\Pr(\{X=1\}) = p,\ \Pr(\{X=0\}) = 1-p$

\item diskret mit $D_X = \begin{cases}
  \{0,1\}  & \text{falls } p \in (0,1) \\
  \{0\}    & \text{falls } p = 0       \\
  \{1\}    & \text{falls } p = 1
  \end{cases}$

\item $F_X(x) = \begin{cases}
  0    & \text{für } x < 0        \\
  1-p  & \text{für } 0 \leq x <1  \\
  1    & \text{für } x \geq 1
  \end{cases}$

\item $\E(X) = p$, $\Var(X) = p-p^2 = p \cdot (1-p)$

\end{itemize}

\textbf{Binomial-Verteilung} $X\sim\mathbf{B}(n,p)$ mit $p \in [0,1]$
\begin{itemize}
\item $\forall k\in\{0,\ldots,n\}\!:
  \Pr(\{X=k\}) = \binom{n}{k} \cdot p^k\cdot (1-p)^{n-k}$

\item diskret mit
  $D_X = \begin{cases}
  \{0,\ldots,n\}  & \text{falls } p \in (0,1)  \\
  \{0\}           & \text{falls } p = 0        \\
  \{n\}           & \text{falls } p = 1
  \end{cases}$

\item $\E(X) = n \cdot p$, $\Var(X) = n \cdot p \cdot (1-p)$

\item Anwendung: Zählen der Erfolge von $n$ unabhängigen, hintereinander
  ausgeführten Experimenten mit Erfolgswahrscheinlichkeit $p$.
\end{itemize}

\textbf{Hypergeometrische Verteilung} $X\sim\mathbf{H}(N,N_0,n)$
\begin{itemize}
\item $\forall l \in D_X\!: \Pr(\{X=l\})
  =\frac{\binom{N_0}{l} \cdot \binom{N-N_0}{n-l}}{\binom{N}{n}}$

\item diskret mit
  $D_X = \{\max(0,n-(N-N_0)),\ldots,\min(N_0,n)\}$

\item Anwendung: unter $N$ Objekten finden sich $N_0$ markierte und es werden
  $n$ entnommen
\end{itemize}

\textbf{Poisson-Verteilung} $X\sim\mathbf{P}(\lambda)$ für $\lambda > 0$
\begin{itemize}
\item $\forall k\in\N_0\!:
  \Pr(\{X=k\}) = \exp(-\lambda) \cdot \frac{\lambda^k}{k!}$

\item diskret mit $D_X = \N_0$

\item $X\sim\mathbf{P}(\lambda_1), Y\sim\mathbf{P}(\lambda_2)
  \Rightarrow X+Y\sim\mathbf{P}(\lambda_1 + \lambda_2)$ \U

\item $\E(X) = \lambda = \Var(X)$, $\E(X^2) = \lambda^2 + \lambda$

\item Anwendung: Approximation von $\mathbf{B}(n,p)$ durch $\mathbf{P}(\lambda)$
  mit $\lambda = n \cdot p$ für 'große' $n$ und 'kleine' $p$, also Eintreffen
  eines seltenen Ereignis bei großer Anzahl an Wiederholungen
\end{itemize}

\newpage
\textbf{Geometrische Verteilung} $X\sim\mathbf{G}(p)$ mit $p \in (0,1]$
\begin{itemize}
\item $\forall k\in\N\!: \Pr(\{X=k\}) = p \cdot (1-p)^{k-1}$

\item diskret mit $D_X = \N$

\item $\E(X) = \frac{1}{p}$, $\Var(X) = \frac{1-p}{p^2}$,
  $\E(X^2) = \frac{2-p}{p^2}$

\item Gedächtnislos:\\ $\forall k_1,k_2\in\N \text{ mit } k_1 < k_2\!:
  \Pr(\{X > k_2\} \mid \{X > k_1\}) = \Pr(\{X > k_2 - k_1\})$

\item Anwendung: diskretes Warten bis zum ersten Eintritt eines Ereignisses
\end{itemize}

\textbf{Gleichverteilung} $X\sim\mathbf{U}([a,b])$ für $-\infty < a < b < \infty$
\begin{itemize}
\item absolut stetig mit
  $f_X(x) = \begin{cases}
  \frac{1}{b-a} 	& \text{falls } x \in [a,b]	\\
  0				& \text{sonst }
  \end{cases}$

\item
  $F_X(x) = \begin{cases}
  0               & \text{falls } x < a	    	\\
  \frac{x-a}{b-a} & \text{falls } x \in [a,b]	\\
  1               & \text{sonst}
  \end{cases}$

\item $\E(X) = \frac{a+b}{2}$, $\Var(X) = \frac{(b-a)^2}{12}$,
  $\E(X^2) = \frac{b^3-a^3}{3(b-a)}$
\end{itemize}

\textbf{Einpunktverteilung} $X\sim\mathbf{U}(\{c\})$ für ein $c\in\R$
\begin{itemize}
\item $P(\{X=c\}) = 1$

\item diskret mit $D_X = \{c\}$

\item
  $F_X(x) = \begin{cases}
  0 	&\text{falls } x < c     \\
  1 & \text{falls } x \geq c
  \end{cases}$

\item $\E(X) = c$, $\Var(X) = 0$
\end{itemize}

\textbf{Exponentialverteilung} $X\sim\mathbf{Exp}(\lambda)$ für $\lambda > 0$
\begin{itemize}
\item absolut stetig mit
  $f_X\!: \R \to [0,\infty],\ x \mapsto \begin{cases}
  \lambda \cdot \exp(-\lambda x) & \text{falls } x \geq 0	\\
  0                              & \text{sonst }
  \end{cases}$

\item
  $F_X(x) = \begin{cases}
  0                    & \text{falls } x \leq 0  \\
  1 - \exp(-\lambda x) & \text{falls } x > 0
  \end{cases}$

\item $\E(X) = \frac{1}{\lambda}$, $\Var(X) = \frac{1}{\lambda^2}$,
  $\E(X^2) = \frac{2}{\lambda^2}$

\item Gedächtnislos: $\forall s,t>0\!: \Pr(\{X>t+s\} \mid \{X>t\}) = \Pr(\{X>s\})$

\item Anwendung: Warten bis zum ersten Eintritt eines Ereignisses
  (zB Lebensdauer, radioaktiver Zerfall,\ldots)
\end{itemize}

\textbf{Standard-Normalverteilung} $X\sim\mathbf{N}(0,1)$
\begin{itemize}
\item absolut stetig mit $f_X\!: \R \to [0,\infty],\
  x \mapsto \frac{1}{\sqrt{2\pi}} \cdot \exp(-\frac{x^2}{2})$

\item $\Phi(x) := F_X(x)=
  \frac{1}{\sqrt{2\pi}} \int_{-\infty}^{x} \exp(-\frac{y^2}{2})~d\lambda(y)$
  (keine explizite Formel)

\item $\forall z\in\R\!: \Phi(z) + \Phi(-z) = 1$ \U

\item $\E(X) = 0$, $\Var(X) = 1$

\end{itemize}

\textbf{Normalverteilung} $X\sim\mathbf{N}(\mu,\sigma^2)$ für
  $\mu,\sigma\in\R,\ \sigma > 0$
\begin{itemize}
\item absolut stetig mit $f_X\!: \R \to [0,\infty],\
  x \mapsto \frac{1}{\sqrt{2\pi\sigma^2}} \cdot \exp(-\frac{(x-\mu)^2)}{2\sigma^2})$

\item $X\sim\mathbf{N}(\mu,\sigma^2)
  \Rightarrow a \cdot X + b \sim\mathbf{N}(a\cdot\mu+b,a^2\cdot\sigma^2)$

\item Also: $\frac{X-\mu}{\sigma} \sim \mathbf{N}(0,1)$

\item $\E(X) = \mu$, $\Var(X) = \sigma^2$
\end{itemize}

\textbf{Gleichverteilung} (mehrdimensional) $X\sim\mathbf{U}(S)$ für $S\subset\R^n$
\underline{endlich}
\begin{itemize}
\item $\forall x\in S\!: \Pr(\{X=x\}) = \frac{1}{|S|}$

\item diskret mit $D_X = S$
\end{itemize}

\textbf{Gleichverteilung} (mehrdimensional) $X\sim\mathbf{U}(G)$ für $G\in\B_d$
\begin{itemize}
\item $\Pr_X(A) = \frac{\lambda_d(A\cap G)}{\lambda_d(G)} =
  \int_A \frac{1}{\lambda_d(G) \cdot \chi_G(x)}~d\lambda(x)$

\item absolut stetig mit $f_X = \frac{1}{\lambda_d(G)} \cdot \chi_G$
\end{itemize}

\textbf{Standard-Normalverteilung} (mehrdimensional)
\begin{itemize}
\item absolut stetig mit
  $f_X\!: \R^d \to [0,\infty),\
  x \mapsto (2\pi)^{-d/2} \cdot \exp(-\frac{1}{2} \sum_{i=1}^{d}x_i^2)$

\item $X = (X_1,\ldots,X_d)$ ist standard-normalverteilt
  $\Leftrightarrow$ $X_1,\ldots,X_d\ \iid \sim\mathbf{N}(0,1)$
\end{itemize}

\section{Sonstiges}

\textbf{Exponentialfunktion}:
\begin{itemize}
\item $\exp\!: \C\to\C,\ x \mapsto \sum_{k=0}^\infty \frac{x^k}{k!}$

\item $\exp(a + b) = \exp(a) \cdot \exp(b)$

\item $\exp$ stetig, $\exp > 0$

\item $\exp' = \exp$
\end{itemize}

\textbf{Natürlicher Logarithmus}
\begin{itemize}
\item $\ln\!: (0,\infty)\to\R:\ x \mapsto \exp^{-1}(x)$

\item $\ln(a \cdot b) = \ln(a) + \ln(b)$

\item $\ln'(x) = \frac{1}{x}$
\end{itemize}

\textbf{Trigonometrische Funktionen}
\begin{itemize}
\item $\cos\!: \C\to\C:\ x \mapsto \sum_{k=0}^\infty (-1)^k\ cdot \frac{x^{2k}}{(2k)!}$

\item $\sin\!: \C\to\C:\ x \mapsto \sum_{k=0}^\infty (-1)^k\ cdot \frac{x^{2k+1}}{(2k+1)!}$

\item $\cos' = -\sin$, $\sin' = \cos$
\end{itemize}

\textbf{Reihen}
\begin{itemize}
\item \textbf{Harmonische Reihe}: $\sum_{n=1}^\infty \frac{1}{n} = \infty$

\item \textbf{Geometrische Reihe}:
  $\sum_{n=0}^\infty a \cdot x^n = \frac{a}{1-x}$
  für $a,x\in\R$ mit $|x| < 1$, divergent für alle anderen $x\in\R$

\item \textbf{Alternierende harmonische Reihe}:
$\sum_{n=1}^\infty (-1)^n \cdot \frac{1}{n} = -\ln(2)$

\item $\sum_{k=0}^\infty \exp(-\lambda) \cdot\ frac{\lambda^k}{k!} = 1$
  für $\lambda > 0$
\end{itemize}

\textbf{Binomischer Lehrsatz:}\\
Für $x,y\in\R,\ n\in\N_0$ gilt:
\[
  (x+y)^n=\sum_{k=0}^n\binom{n}{k}x^{n-k}y^k
\]
s

\end{document}
