\documentclass[11pt,a4paper,ngerman]{article}

% BASICS
\usepackage[ngerman]{babel}
\usepackage[utf8x]{inputenc}
\usepackage{parskip}
\usepackage{amsfonts}
\usepackage{amsmath}
\usepackage{amssymb}
\usepackage{enumerate}
\usepackage{float}
\usepackage{graphicx}
\usepackage{caption}
\usepackage{multicol}
\usepackage[left=3cm,right=3cm,top=2cm,bottom=3.5cm,includeheadfoot]{geometry}
\usepackage{url}
\usepackage{hyperref}

% ZAHLEN
\newcommand{\N}{\mathbb{N}}
\newcommand{\Z}{\mathbb{Z}}
\newcommand{\Q}{\mathbb{Q}}
\newcommand{\R}{\mathbb{R}}
\newcommand{\C}{\mathbb{C}}

% MENGEN
\newcommand{\Po}{\mathfrak{P}}
\newcommand{\compl}{^{\text{\tiny C}}}
\newcommand{\pd}{\underline{\operatorname{p.d.}}}

% STOCHASTIK
\renewcommand{\O}{\Omega}
\newcommand{\A}{\mathcal{A}}
\renewcommand{\Pr}{\mathbb{P}}
\newcommand{\E}{\mathbb{E}}
\newcommand{\Var}{\operatorname{Var}}
\newcommand{\Cov}{\operatorname{Cov}}
\newcommand{\F}{\mathcal{F}}
\newcommand{\e}{\mathcal{E}}
\renewcommand{\L}{\mathcal{L}}
\newcommand{\B}{\mathfrak{B}}
\renewcommand{\o}{\omega}
\newcommand{\iid}{\operatorname{i.i.d}}

% ANALYSIS
\newcommand{\limfty}[1]{\lim\limits_{#1\to\infty}}
\renewcommand{\max}{\operatorname{max}}
\renewcommand{\min}{\operatorname{min}}

% SONSTIGES
\newcommand{\U}{\texttt{Übung}}

% CENTERED COLUMNS WITH FIX WIDTH
\usepackage{array}
\newcolumntype{P}[1]{>{\centering\arraybackslash}p{#1}}

% META
\date{April 2019}
\author{Maximilian Reif}
\title{Zusammenfassung Einführung in die Stochastik}
\hypersetup{pdftex,
            pdfauthor={Maximilian Reif},
            pdftitle={Zusammenfassung Einführung in die Stochastik},
            pdfsubject={},
            pdfkeywords={},
            pdfproducer={},
            pdfcreator={},
            pdfpagemode=UseOutlines
}

%-------------------------------------------------------------------------------

\begin{document}

\begin{titlepage}
    \ \newline\newline\newline\newline\newline

  \begin{center}

  \huge Zusammenfassung zur\\
  \Huge\textbf{Einführung in die Stochastik}\\
  \huge im Wintersemester 17/18\\
  \normalsize

  \vspace{1cm}
  \begin{tabular}[b]{l|l}
  \textbf{Autor}         & Maximilian Reif
    \texttt{\href{mailto:reifmaxi@fim.uni-passau.de}{<reifmaxi@fim.uni-passau.de>}} \\
  \textbf{Version}       & April 2019 \\
  \textbf{GitHub}        & \url{https://github.com/lordreif/stochastik}
  \end{tabular}
  \vspace{1cm}

  \end{center}

  \tableofcontents

  % FIM logo
  \begin{figure}[b]
  \centering
  %\includegraphics[width=0.6\textwidth]{/home/lordreif/Templates/img/fim.png}
  \end{figure}


\end{titlepage}

\setcounter{page}{1}
\include{sections/grundlagen}
\include{sections/integration}
\include{sections/zufallsvariablen}
\include{sections/erwartungswert_varianz}
\include{sections/verteilungen}
\include{sections/sonstiges}

\end{document}
