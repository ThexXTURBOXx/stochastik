\section{Integration}

\textbf{Lebesgue-Maß} $\lambda_d\!: \B_d \to [0,\infty]$
\begin{itemize}
\item Für $B_1,B_2,\ldots\in\B_d\ \pd$:\:
  $\lambda_d(\bigcup_{i=1}^{\infty} B_i) = \sum_{i=1}^{\infty} \lambda_d(B_i)$.

\item Für $a_1,b_1,\ldots,a_d,b_d\in\R$ mit $a_i \leq b_i$:\:\:
  \mbox{$\lambda_d([a_1,b_1] \times\cdots\times [a_d,b_d]) =
  \prod_{i=1}^{d} (b_i-a_i)$.}

\item Für $a\in\R^d$, $Q\in\mathrm{O}(d)$, $A\in\B_d$:
  $\lambda_d(a+Q(A)) = \lambda_d(A)$.

\item $\forall x\in\R^d\!: \lambda_d(\{x\}) = 0$

\item $\lambda_d$ ist $\sigma$-subadditiv, $\sigma$-stetig von unten
  (nicht von oben!) und monoton
\end{itemize}

\textbf{Borel-messbare Funktionen}
\begin{itemize}
\item $f\!: \R^d\to\R$ \textit{Borel-messbar}
  $:\Leftrightarrow \forall B\in\B_1\!: \{f \in B\} \in\B_d$

\item $f\!: \R^d\to\overline{\R}$ \textit{Borel-messbar}
  $:\Leftrightarrow \forall B\in\B_1\!: \{f \in B\} \in\B_d$
  \mbox{und $\{f = \infty\} \in\B_d$}

\item $\F_d \coloneqq \{f\!: \R^d \to \overline{\R} \mid f \text{ ist Borel-messbar}\}$,
  $\F_d^+ \coloneqq \{f\in\F_d \mid f \geq 0\}$

\item $\forall B\in\B_d\!: \mathds{1}_B=\chi_B\in\F_d^+$

\item $f$ stetig $\Rightarrow f\in\F_d$

\item $f,g\in\F_d \Rightarrow a \cdot f + g, f \cdot g, f/g,
  \max(f,g)\in\F_d$ \mbox{falls wohldefiniert ($a\in\overline{\R}$)}
\end{itemize}

\textbf{Elementare Funktionen}
\begin{itemize}

\item Abbildung $f\!: \R^d \to \overline{\R}$ \textit{elementar} $:\Leftrightarrow$\\
  $\exists b_1,\ldots,b_n\in\overline{\R} \wedge \exists B_1,\ldots,B_n\in\B_d\
  \pd$ mit $f(x) = \sum_{i=1}^{n} b_i \cdot \chi_{B_i}(x)$ für alle $x\in\R^d$

\item $\e_d \coloneqq \{f\!: \R^d \to \overline{\R} \mid f \text{ ist elementar}\}$,
  $\e_d^+ \coloneqq \{f\in\e_d \mid f\geq 0\}$

\item $\e_d \subset \F_d$
\end{itemize}

\textbf{Lebesgue-Integral}
\begin{itemize}
\item Für $f\in\e_d^+$:
  $\int f~d\lambda_d \coloneqq \sum_{i=1}^{n} b_i \cdot \lambda_d(B_i) \in [0,\infty]$

\item Das Lebesgue-Integral ist definiert für alle $f\in\F_d^+$.

\item $\int_B f~\lambda_d \coloneqq \int (f\cdot \chi_B)~d\lambda_d$ für $B\in\B_d$

\item Das Lebesgue-Integral ist monoton und linear.

\item 4.28: Für auf $[a,b]$ beschränktes $f \in \F_1$ mit
  \mbox{$\lambda(\{x\in[a,b] \mid f \text{ unstetig in }x\}) = 0$ gilt:}
  \[
    \int_{[a,b]} f~d\lambda =
    \underbrace{\int_{a}^{b}f(x)~dx}_{\text{Riemann-Integral}}
  \]

\item $\forall A\in\B_d\!: \int_A f~d\lambda_d = \int_A g~d\lambda_d$
  $\Leftrightarrow \lambda_d(\{x\in\R^d \mid f(x) \neq g(x)\}) = 0$
\end{itemize}

\textbf{Satz von der monotonen Konvergenz}\\
Für jede monoton wachsende Folge $0 \leq f_1 \leq f_2 \leq \cdots$
in $\F_d^+$ gilt
\[
  \limfty{i}\int f_i~d\lambda_d =
  \int\limfty{i} f_i~d\lambda_d \in [0,\infty].
\]

\textbf{Satz von der dominierten Konvergenz}\\
Für $f,f_1,f_2,\ldots\in\F_d$ mit $\limfty{i} f_i = f$
und $g\in\F_d$ integrierbar mit $|f_i| \leq g$ für alle $i\in\N$ gilt
\[
  \int f~d\lambda_d = \limfty{i} \int f_i~d\lambda_d.
\]
Insbesondere für $f\in\F_1$ integrierbar oder $f\in\F_1^+$:
\[
  \int_\R f~d\lambda = \limfty{i} \int_{[-i,i]} f~d\lambda.
\]


\textbf{Satz von Fubini}\\
Für $f\in\F_d$ integrierbar oder $f\in\F_d^+$ und
$B = B_1 \times\cdots\times B_d \subset \B_1^d$ gilt
\[
  \int_B f~d\lambda_d = \int_{B_{i_1}} \cdots \int_{B_{i_d}} f(x_1,\ldots,x_d)~
  d\lambda(x_{i_d}) \cdots d\lambda(x_{i_1})
\]
für jede Permutation $(i_1,\ldots,i_d)$ von $(1,\ldots,d)$.

\textbf{Substitutionsregel} (Riemann-Integration)\\
Für $a,b\in\R$ mit $a<b$, $g\!: [a,b] \to I$ stetig differenzierbar,
$f\!: I\to\R$ stetig gilt
\[
  \int_a^b f(g(y)) \cdot g'(y)~dy = \int_{g(a)}^{g(b)}f(x)~dx.
\]

\textbf{Partielle Integration} (Riemann-Integration)\\
Für $a,b\in\R$ mit $a<b$ und $f,g\!: [a,b]\to\R$ stetig differenzierbar gilt
\[
  \int_a^b f'(x) \cdot g(x)~dx
  = f(x) \cdot g(x) \Big\vert_{x=a}^b - \int_a^b f(x) \cdot g'(x)~dx.
\]
